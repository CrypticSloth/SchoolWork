\documentclass[]{article}
\usepackage{lmodern}
\usepackage{amssymb,amsmath}
\usepackage{ifxetex,ifluatex}
\usepackage{fixltx2e} % provides \textsubscript
\ifnum 0\ifxetex 1\fi\ifluatex 1\fi=0 % if pdftex
  \usepackage[T1]{fontenc}
  \usepackage[utf8]{inputenc}
\else % if luatex or xelatex
  \ifxetex
    \usepackage{mathspec}
  \else
    \usepackage{fontspec}
  \fi
  \defaultfontfeatures{Ligatures=TeX,Scale=MatchLowercase}
\fi
% use upquote if available, for straight quotes in verbatim environments
\IfFileExists{upquote.sty}{\usepackage{upquote}}{}
% use microtype if available
\IfFileExists{microtype.sty}{%
\usepackage[]{microtype}
\UseMicrotypeSet[protrusion]{basicmath} % disable protrusion for tt fonts
}{}
\PassOptionsToPackage{hyphens}{url} % url is loaded by hyperref
\usepackage[unicode=true]{hyperref}
\hypersetup{
            pdfborder={0 0 0},
            breaklinks=true}
\urlstyle{same}  % don't use monospace font for urls
\IfFileExists{parskip.sty}{%
\usepackage{parskip}
}{% else
\setlength{\parindent}{0pt}
\setlength{\parskip}{6pt plus 2pt minus 1pt}
}
\setlength{\emergencystretch}{3em}  % prevent overfull lines
\providecommand{\tightlist}{%
  \setlength{\itemsep}{0pt}\setlength{\parskip}{0pt}}
\setcounter{secnumdepth}{0}
% Redefines (sub)paragraphs to behave more like sections
\ifx\paragraph\undefined\else
\let\oldparagraph\paragraph
\renewcommand{\paragraph}[1]{\oldparagraph{#1}\mbox{}}
\fi
\ifx\subparagraph\undefined\else
\let\oldsubparagraph\subparagraph
\renewcommand{\subparagraph}[1]{\oldsubparagraph{#1}\mbox{}}
\fi

% set default figure placement to htbp
\makeatletter
\def\fps@figure{htbp}
\makeatother


\date{}

\begin{document}

\section{Quantum Neural Network for Reinforcement
Learning}\label{quantum-neural-network-for-reinforcement-learning}

\paragraph{Erik Sorensen}\label{erik-sorensen}

\paragraph{Data Science Honors Project
Proposal}\label{data-science-honors-project-proposal}

\subsection{Introduction}\label{introduction}

Quantum computing and machine learning are two exciting new fields.
Quantum computing brings a new perspective on computers by using
properties of quantum mechanics to do computations, which can be
exponentially faster than the classical computers we know today. In
fact, some tasks that we thought were impossible with computers today,
can be accomplished using quantum computers. (CITATION) Machine learning
is another exciting field that learns complex problems from large
amounts of data that are too difficult for humans to manually solve. It
been used in facial recognition (CITATION) and identifying weeds in
precision agriculture (CITATION). A more recent sub-field of machine
learning is \emph{reinforcement learning}. Reinforcement learning uses a
positive or negative reinforcement signal from an environment, known as
the \emph{reward}, to provide feedback to the learning system so that it
may learn and improve to maximize this reward. Some fields that use
reinforcement learning are in economics to artificial intelligence in
games. (CITATION) Reinforcement learning is an attractive tool to use.
While other machine learning techniques often times require a well
tuned, large dataset to learn, reinforcement learning only requires a
reward from an environment. This provides substantial control over the
results of learning and saves time on finding and tuning a large
dataset. Even further, we can use new quantum computing techniques to
exponentially increase the power of these reinforcement learning
techniques.

\end{document}
